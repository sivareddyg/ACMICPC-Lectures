\chapter{定界搜索}

\section{组合优化问题}

在归约一章中,我们介绍了著名的NP-完全问题:CNF-SAT。SAT是大量问题的一般形式,因此在实际中被广泛应用。然而SAT solver也有其局限:
SAT是一个判定问题,当我们试图把很多常见的问题如最大团(Clique)、最小集合覆盖(Set Cover)等问题归约到判定版本的SAT时,困难就出现了。
这些问题都有共同的形式:它们都属于组合优化问题。组合优化问题通常是要在一个离散的集合中求解满足一定约束条件的最优解,而不仅仅是判定满足约束条件的解是否存在。

相应的,SAT问题也有其推广版本:带权Max-SAT。如果我们给SAT问题中的每个布尔变量$x_i$一个权值$W_i$,并且求满足$C = C_1 \land C_2 \land \ldots \land C_n$赋值
中权值最大的那个:
$$
\begin{array}{cl}
\displaystyle \argmax_{\mathbf{x}} & \sum_{i\in V} W_i, \textrm{其中\,}V = \{i~|~x_i = T\} \\
\textrm{subject to} & C(\mathbf{x}) = T \\
\end{array}
$$

可想而知,带权Max-SAT问题的求解比SAT要困难得多,相应能够求解问题的规模也比SAT solver小得多。
所以,当我们面对NP-难的优化问题时,机械地把它归约到带权Max-SAT有时并不能满足我们的需求。我们需要从问题自身的结构出发
设计有针对性的求解算法。

遗传算法和模拟退火算法常用于约束条件不强而解空间非常大的搜索问题。这类算法的特点是通常可以很短的时间内就得到一个相当好的解,
但有时难以回答何时收敛以及是否收敛到全局最优解这个问题。它们的应用场景通常是并不需要精确最优解的规模巨大的组合优化难问题。

\section{搜索与图遍历}

\section{设计启发函数}

如果我们有一个近似比为$\alpha$的近似算法,求得一个解为$X$。根据近似算法的性质,我们知道$X \leq \alpha OPT$,
也就得到原问题的一个下界$$OPT\geq \frac{1}{\alpha} X.$$
知道$2MST$是最短哈密顿路的$2$-近似,所以
$$2MST\leq 2OPT, MST\leq OPT$$
是一个对哈密顿路的下界估计。

将近似算法作为下界的估计有一个问题:很多近似算法的性能非常之好,以至于在实际情况下与最优解相差无几。一旦除以近似比,下界的估值就显得相当小了。
而为了体现启发函数的作用,我们希望下界估计得尽可能大。为不同的问题设计足够紧的下界,

\begin{prob}[比较排序]
 我们知道任何基于比较的排序算法,为了排序5个数,在最坏情况下都需要7次比较。
 如果我们给定$n$个数(假设它们互不相同),并且已知其中一些的大小关系(例如第3个数大于第4个数),
 问将它们排序至少还需要比较多少次?
\end{prob}
\begin{solution}
 
\end{solution}
