\usepackage{amssymb}
\usepackage{amsthm}
\usepackage{amsmath}
\usepackage{fontspec,xunicode,xltxtra}
\usepackage{titlesec}
\usepackage{indentfirst}
\usepackage[BoldFont,SlantFont]{xeCJK}
\usepackage{fancyhdr}
\usepackage{graphicx}
\usepackage{listings}
\usepackage{multicol}
\usepackage{tikz}
\usetikzlibrary{matrix,fit}
% \XeTeXinputencoding "cp936"

\setCJKfamilyfont{kai}{KaiTi_GB2312}
\setCJKfamilyfont{hei}{STXihei}
\setCJKmainfont{SimSun}
%\setromanfont{Liberation Serif}

\pagestyle{fancy}
%\rhead{{\sf\thepage}}
%\lhead{\kai ACM/ICPC Code Library}

%\fancyhead[L]{\sf\kai \leftmark} 
%\fancyhead[R]{\sf\kai \rightmark}

\newcommand{\kai}{\CJKfamily{kai}}
\newcommand{\hei}{\CJKfamily{hei}}

\setlength{\parindent}{2.2em}

\renewcommand{\contentsname}{\hei 目录}


% settings for listings
\lstset {
  basicstyle = \small\monotype,
  language = C++,
  tabsize = 2,
  breaklines = true,
  breakindent = 1.1em,
  numbers=right,
  stringstyle=\monotype,
  numberstyle=\footnotesize\ttfamily,
  firstnumber=last,
  basewidth={0.55em, 0.5em}
}

% font of section header
\usepackage{sectsty}
\allsectionsfont{\hei}

% chapter format
\titleformat{\chapter}{\huge\bfseries\hei}{第\,\thechapter\,章}{1em}{}

% main font of code, Liberation Mono looks pretty good
\newfontfamily{\monotype}{Liberation Mono}


\renewcommand{\contentsname}{目录}
\renewcommand{\listfigurename}{图目录}
\renewcommand{\listtablename}{表目录}
\renewcommand{\partname}{第 \thepart 部分}
%\renewcommand{\chaptername}{第 \thechapter 章} %%??
\renewcommand{\figurename}{图}
\renewcommand{\tablename}{表}
\renewcommand{\bibname}{参考文献}
\renewcommand{\appendixname}{附录}
\renewcommand{\indexname}{索引}
%\renewcommand{\abstractname}{摘要}
%\renewcommand{\refname}{参考文献}


\newtheorem{prob}{问题}

\tikzstyle{vertex} = [circle,draw,fill=black!15,minimum size=20pt,inner sep=0pt]



\usepackage{enumitem}
\setenumerate[1]{itemsep=3pt,partopsep=0pt,parsep=\parskip,topsep=5pt}
\setitemize[1]{itemsep=3pt,partopsep=0pt,parsep=\parskip,topsep=5pt}
\setdescription{itemsep=3pt,partopsep=0pt,parsep=\parskip,topsep=5pt}


\DeclareMathOperator*{\argmax}{arg\,max}


\newenvironment{solution}{}{\medskip}